\section{UNDERSTANDING SPAN, SUBSPACE}

\begin{enumerate}
	
	\item \textbf{Solution:}
	\begin{enumerate}
	\item Firstly, we show that $span\{\mathcal{S}\}\subseteq\mathcal{M}$. For $\forall \vec{x}\in span\{\mathcal{S}\}$
	$$\vec{x}=\sum_{i=1}^{n}\lambda_i\vec{v}_i$$
	By the definition of $\mathcal{V}$ that $\mathcal{V}$  denotes the subspace containing $\mathcal{S}$, for 
	$\forall \mathcal{V}$ we have 
	$$\vec{x}=\sum_{i=1}^{n}\lambda_i\vec{v}_i\in \mathcal{V}$$
	Thus, for the intersection of all $\mathcal{V}$ $\mathcal{M}$ we have
	$$\vec{x}\in \cap_{\mathcal{S}\subseteq\mathcal{V}}\mathcal{V}=\mathcal{M}$$
	Secondly, we show that $\mathcal{M}\subseteq span\{\mathcal{S}\}$. For $\forall \vec{x}\in \mathcal{M}$, since $span\{\mathcal{S}\}$ itself is a satisfied $\mathcal{M}$, it follows that $\forall \vec{x} \in \mathcal{M} \in span\{\mathcal{S}\}$. Proved.\\
	In conclusion, $span\{\mathcal{S}\}\subseteq\mathcal{M}, \mathcal{M}\subseteq span\{\mathcal{S}\}$ implies that $span\{\mathcal{S}\}=\mathcal{M}$. Q.E.D.
	
	\end{enumerate}

\end{enumerate}